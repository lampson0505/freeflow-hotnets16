\section{System to Build} \label{sec:promise}

%P1 introduce the two components we want to build
\Name's main components will include a container locator and a virtualized NIC.
The container locator can figure out where the location of the container is and
decide the most efficient way for two containers to talk with each other (e.g. 
via shared memory, rdma or dpdk).
And the virtualized NIC emulates the necessary underlying resource structures 
(e.g., send queue or receive queue of the NIC). In this way, the application will
be using the standardized APIs without be aware of the actual communication
mechanisms. The architecture of \Name is shown in Figure~\ref{fig:system_modules}.

     \begin{figure}[ht]
     \centering 
     \includegraphics[width=0.5\textwidth]{figures/system/system_modules.pdf}      
     \label{fig:system_modules}
     \caption{The system architecture of \Name.} 
     \end{figure}

\subsection{Container Locator}
The Container Locator is a logic module we will add inside the communication
library of a container's application. For example, if the application is a RDMA application,
the Container Locator will be inside library \texttt{libibverbs}. 

     \begin{figure}[ht]
     \centering 
     \includegraphics[width=0.45\textwidth]{figures/system/system_locator.pdf}      
     \label{fig:system_locator}
     \caption{The logic flow of Container Locator.} 
     \end{figure}

The Container Locator send queries to Mesos and the cloud's Fabric Controller to locate
the sender/receiver containers and decide the most efficient way they talk with each other.
The logic flow of this process is as shown in Figure~\ref{fig:system_locator}.
First the container 

%We are going to add a logic module called Container Locator inside the application's
%library for communication. For example, the  



\subsection{Virtualized NIC}


%\begin{itemize}
%  \item Container Locator:
%  \item Virtualized NIC:
%\end{itemize}

